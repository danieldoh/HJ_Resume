\section{Research Experience}
%====================
% 3D Hand and Object Interaction 
%====================
\subsection{{\href{https://engineering.purdue.edu/cdesign/wp/}{Convergence Design Lab} \hfill January 2023 – January 2024 }}
\subtext{Undergraduate Research Assistant, Advisor: Dr. Karthik Ramani \hfill Purdue University, West Lafayette} 
\subsubsection*{\footnotesize An Exploratory Study on Multi-modal Generative AI in AR Storytelling (Submitted to CSCW 2024) \hfill Lead} 
\vspace{-0.5em}
\begin{itemize}
    \setlength\itemsep{-0.5em}
    %\addtolength{\leftskip}{0.2in}
    \item \small Led research on AI-generated multi-modal content's impact on AR Storytelling creation and perception by an exploratory study.
    % \item Summarized a design space of multi-modal AR Storytelling and implemented a cognitive model for understanding the roles of authors and audiences in the storytelling process.
    % \item Developed an experimental AR Storytelling platform with AI-generated multi-modal content, integrating with multiple state-of-the art generative AI models.
    %\item Investigated the impact of AI-generated multi-modal content on the creation and perception of AR Storytelling through an exploratory study.
\end{itemize}

\vspace{-2em}
\subsubsection*{\footnotesize An HCI-Centric Survey and Taxonomy of Human-Generative-AI Interactions (Submitted to CHI 2024) [1] \hfill Co-Author}
\vspace{-0.5em}
\begin{itemize}
    \setlength\itemsep{-0.5em}
    %\addtolength{\leftskip}{0.2in}
    \item \small Reviewed 154 papers on Generative AI applications and contributed to synthesizing a taxonomy of human-GenAI interactions.
    %\item Submitted to CHI 2024 
    %\item Analyzed the recent advancements and research in the domain of GenAI-based systems.
    %\item Contributed to synthesizing a taxonomy of human-GenAI interactions for future design in the field.
\end{itemize}

\vspace{-2em}
\subsubsection*{\footnotesize Visualizing Causality in Mixed Reality for Manual Task Learning: Exploratory Study (Submitted to CHI 2024) [2] \hfill Co-Author
}
\vspace{-0.5em}
\begin{itemize}
    \setlength\itemsep{-0.5em}
    %\addtolength{\leftskip}{0.2in}
    \item \small Implemented a MR interface for learning assembly tasks with visual representations of causal relationships.
    %\item Submitted to CHI 2024 
\end{itemize}

\vspace{-2em}
\subsubsection*{\footnotesize CARING-AI: Towards Authoring Context-aware Augmented Reality \hfill Co-Author \\ INstruction through Generative Artificial Intelligence (Submitted to CHI 2024)}
\vspace{-0.5em}
\begin{itemize}
    \setlength\itemsep{-0.5em}
    %\addtolength{\leftskip}{0.2in}
    \item \small Implemented AR interface for authoring instructions and developed a user interface to evaluate the system via a user study.
    %\item Submitted to CHI 2024 %based on textual input describing tasks, avatars' trajectory, and directional vision.
    %\item Contributed by developing various application scenarios for this system in diverse contexts.
    %\item Developed a user interface for a user study and evaluated the system.
\end{itemize}

%\item Developed a web browser user interface using JavaScript and an API for prompt engineering purposes.
% \item Led research on immersive storytelling using Augmented Reality (AR) and Generative AI.
% \item Contributed to research on context-aware AR instruction through Generative AI, with a focus on developing AR user interfaces.
% \item Participated in an HCI-centric survey concerning the taxonomy of Human-Generative-AI interactions.

%\item Implemented an Augmented Reality user interface using Unity to interact with 3D human motion generated by AI.
%\item Preformed 3D reconstructions of Hand and Object meshes using MeshLab, focusing on capturing spatial information.
%\item Participated in the development of a Human-Computer Interaction project
%\item Contributed as a co-author to the "CARING-AI" paper, submitted for proceedings at CHI 2024, involving the development of Unity and web browser user interfaces (under review)
%Contributed to a Human-Computer Interaction project, actively participating in its development
%\item Computed pose estimation of objects and determined their positions/orientations using computational methods
%====================
% SURF
%====================
\subsection{{Summer Undergraduate Research Fellowship (SURF) \hfill May 2023 – August 2023}}
\subtext{Engineering Undergraduate Research Office (EURO), Advisor: Dr. Karthik Ramani \hfill Purdue University, West Lafayette}
\begin{zitemize}
\item Conducted research on animation generation from text and human pose by combining two diffusion models.
%\item Authored a technical paper summarizing the findings from the SURF research.
\item Presented a poster at the SURF Symposium, effectively conveying the key insights and contributions of the research.
\end{zitemize}


%====================
% VIP Smart Cities 
%====================
\subsection{{Vertically Integrated Projects \hfill August 2022 – May 2023}}
\subtext{VIP Smart Cities Research Team, Advisor: Dr. Mohammad Reza Jahanshahi \hfill Purdue University, West Lafayette}
\begin{zitemize}
\item Implemented a semantic segmentation neural network to detect cracks and scratches.
%\item Constructed a neural network for Image Classification, demonstrating proficiency in deep learning techniques.
\item Generated 56 cracks and scratches datasets, each comprising 308 images, by utilizing the Houdini program.
%\item Skillfully implemented an OpenAI Gym Environment, enabling reinforcement learning experimentation

\end{zitemize}

%====================
% Package Manager
%====================
%\subsection{{Package Manager Application Programming Interface (API) \hfill January 2023 – May 2023}}
%\subtext{Software Engineering\hfill Purdue University, West Lafayette}
%\begin{zitemize}
%\item Implemented 17 features, totaling 2029 lines of code, to manage project packages using TypeScript and Firebase
%\item Set up Continuous Integration (CI) and Continuous Delivery (CD) pipelines for efficient development workflows.
%\item Conducted security analysis to ensure API robustness and protect against vulnerabilities.

%\end{zitemize}
